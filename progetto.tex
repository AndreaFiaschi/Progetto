\documentclass[a4paper,11pt,english]{article}
\usepackage[english]{babel}
\usepackage[utf8]{inputenc}
\usepackage{graphicx}
\usepackage{mdframed}
\usepackage{wrapfig}
\usepackage{enumitem}
\usepackage{amsmath}
\usepackage{amssymb}
\usepackage{appendix}
\usepackage{amscd}
\usepackage{subfig}
\usepackage{cite}
\usepackage[autostyle,italian=guillemets]{csquotes}
\usepackage{bm}
\usepackage{vmargin}
\usepackage{hyperref}

\newmdenv[
  topline=false,
  bottomline=false,
  rightline=false,
  skipabove=\topsep,
  skipbelow=\topsep,
  leftmargin=-10pt,
  rightmargin=0pt,
  innertopmargin=0pt,
  innerbottommargin=0pt
]{siderules}

\setmarginsrb{20mm}{20mm}{20mm}{20mm}{0pt}{0mm}{0pt}{0mm}
\hyphenation{ShuntLDO}
\hyphenation{DC-DC}
\hyphenation{HL-LHC}

\begin{document}
%
%---- SLASH
\def\slasha#1{#1\hskip-0.65em /}  %slasha per caratteri piccoli
\def\slashb#1{#1\hskip-1.3em /}   %slashb per quelli grandi
\def\slashc#1{#1\hskip-.4em /}
%
%---- UNITA` DI MISURA
\def \pb        {{\rm \, pb}}
\def \ipb       {{\rm \, pb^{-1}}}
\def \ifb       {{\rm \, fb^{-1}}}
\def \eV        {{\rm \,  e\kern-0.125em V}}
\def \keV       {{\rm \, ke\kern-0.125em V}}
\def \MeV       {{\rm \, Me\kern-0.125em V}}
\def \GeV       {{\rm \, Ge\kern-0.125em V}}
\def \TeV       {{\rm \, Te\kern-0.125em V}}
\def \Hz        {{\rm \, Hz}}
\def \kHz       {{\rm \, kHz}}
\def \MHz       {{\rm \, MHz}}
\def \GHz       {{\rm \, GHz}}
\def \Ohm       {{\rm \, \Omega}}
\def \mOhm       {{\rm \, m\Omega}}
\def \kOhm       {{\rm \, k\Omega}}
\def \MGy       {{\rm \, MGy}}
\def \Grad      {{\rm \, Grad}}
\def \TeVc      {\TeV\kern-0.125em /c}
\def \TeVcc     {\TeV\kern-0.125em /c^2}
\def \GeVc      {\GeV\kern-0.125em /c}
\def \GeVcc     {\GeV\kern-0.125em /c^2}
\def \MeVc      {\MeV\kern-0.125em /c}
\def \MeVcc     {\MeV\kern-0.125em /c^2}
\def \modud     {{1$\times$2}}
\def \moddd     {{2$\times$2}}

%
%---- SIMBOLI
\def\ga{\mathrel{\raise.3ex\hbox{$>$\kern-.75em\lower1ex\hbox{$\sim$}}}}
\def\la{\mathrel{\raise.3ex\hbox{$<$\kern-.75em\lower1ex\hbox{$\sim$}}}}
%\newcommand {\lesssim}
%     {\,\raisebox{-0.6ex}{$\stackrel{\textstyle<}{\textstyle\sim}$}\,}
%\newcommand {\gtrsim}
%     {\,\raisebox{-0.6ex}{$\stackrel{\textstyle>}{\textstyle\sim}$}\,}
\newcommand{\ckm}{$\checkmark$}
%
%---- MISCELLANEA
\newcommand {\slashed}[1] { \mbox{\rlap{\hbox{/}} #1 }}
\newcommand {\onehalf}    {\raisebox{0.1ex}{${\frac{1}{2}}$}}
\newcommand {\fivethirds} {\raisebox{0.1ex}{${\frac{5}{3}}$}}
\newcommand {\OR}         {{\tt OR}\,}
\newcommand {\rts}        {\sqrt{s}}
\newcommand {\lumi}       {\mathcal{L}}
\newcommand {\Lumi}       {\int\lumi\mathrm{d}t}
\newcommand {\degrees}    {^\circ}
\newcommand {\VDD}        {\mathrm{V_{out}}}
\newcommand {\VDDD}       {\mathrm{V_{outD}}}
\newcommand {\VDDA}       {\mathrm{V_{outA}}}
\newcommand {\Iin}        {\mathrm{I_{in}}}
\newcommand {\Vin}        {\mathrm{V_{in}}}
\newcommand {\Iload}      {\mathrm{I_{load}}}
\newcommand {\de}         {\partial}
\newcommand {\uA}         {\; \mu \rm A}
\newcommand {\um}         {\; \mu \rm m}
\newcommand {\nm}         {\rm \; nm}
\newcommand {\ms}         {\; \rm m \rm s}
\newcommand {\us}         {\; \mu \rm s}
\newcommand {\cm}         {\rm \; cm}
\newcommand {\mm}         {\rm \; mm}
\newcommand {\mV}         {\rm \; mV}
\newcommand {\mA}         {\rm \; mA}
\newcommand {\m}          {\rm \; m}
\newcommand {\km}         {\rm \; km}
\newcommand {\bx}         {\rm \; BX}
\newcommand {\A}          {\rm \, A}
\newcommand {\V}          {\rm \; V}
\newcommand {\T}          {\rm \; T}
\newcommand {\kV}         {\rm \; kV}
\newcommand {\pF}         {\rm \; pF}
\newcommand {\kW}         {\rm \; kW}
\newcommand {\kA}         {\rm \; kA}
\newcommand {\kVm}        {\rm \; kV\! / \! m} 
\newcommand {\MVm}        {\rm \; MV\! / \! m} 
\newcommand {\ns}         {\rm \; ns}
 
%
%---- THEORY groups & AOB
\newcommand {\gws}        {\mathrm{SU(2)_L \otimes U(1)_Y}}
\newcommand {\sul}        {\mathrm{SU(2)_L}}
\newcommand {\suc}        {\mathrm{SU(3)_C}}
\newcommand {\ul}         {\mathrm{U(1)_Y}}
\newcommand {\uem}        {\mathrm{U(1)_{em}}}
\newcommand {\sigmabar}   {\overline{\sigma}}
\newcommand {\gmunu}      {g^{\mu \nu}}
\newcommand {\munu}       {{\mu \nu}}
\newcommand {\obra}       {\langle 0 |}
\newcommand {\oket}       {| 0 \rangle}
%
%---- THEORY lepton fields
\newcommand {\LL}         {L^{\alpha}_{\mathrm L}}
\newcommand {\LLd}        {L^{\dagger \alpha}_{\mathrm L}}
\newcommand {\lL}         {\ell^{\alpha}_{\mathrm L}}
\newcommand {\lLd}        {\ell^{\dagger \alpha}_{\mathrm L}}
\newcommand {\ld}         {\ell^{\dagger \alpha}}
\newcommand {\lb}         {\overline{\ell}^{\alpha}}
\newcommand {\lR}         {\ell^{\alpha}_{\mathrm R}}
\newcommand {\lRd}        {\ell^{\dagger \alpha}_{\mathrm R}}
\newcommand {\nuL}        {\nu^{\alpha}_{\mathrm L}}
\newcommand {\nuLb}       {\overline{\nu}^{\alpha}_{\mathrm L}}
\newcommand {\nub}        {\overline{\nu}^{\alpha}}
\newcommand {\lept}       {\ell^\alpha}
\newcommand {\neut}       {\nu^{\alpha}}
\newcommand {\nuLd}       {\nu^{\dagger \alpha}_{\mathrm L}}
\newcommand {\Phid}       {\Phi^\dagger}
%
%---- THEORY quark fields
\newcommand {\up}         {u^{\alpha}}
\newcommand {\ub}         {\overline{u}^{\alpha}}
\newcommand {\down}       {d^{\alpha}}
\newcommand {\db}         {\overline{d}^{\alpha}}
\newcommand {\QL}         {Q^{\alpha}_{\mathrm L}}
\newcommand {\QLd}        {Q^{\dagger \alpha}_{\mathrm L}}
\newcommand {\UL}         {U^{\alpha}_{\mathrm L}}
\newcommand {\ULd}        {U^{\dagger \alpha}_{\mathrm L}}
\newcommand {\UR}         {U^{\alpha}_{\mathrm R}}
\newcommand {\URd}        {U^{\dagger \alpha}_{\mathrm R}}
\newcommand {\DL}         {D^{\alpha}_{\mathrm L}}
\newcommand {\DLd}        {D^{\dagger \alpha}_{\mathrm L}}
\newcommand {\DR}         {D^{\alpha}_{\mathrm R}}
\newcommand {\DRd}        {D^{\dagger \alpha}_{\mathrm R}}
\newcommand {\bfell}      {\ell\kern-0.4em
                           \ell\kern-0.4em
                           \ell\kern-0.4em
                           \ell }
\newcommand {\obfell}     {\overline{\ell}\kern-0.4em
                           \overline{\ell}\kern-0.4em
                           \overline{\ell}\kern-0.4em
                           \overline{\ell}}
\newcommand {\bfH}      {\; {\cal H}\kern-0.5em \kern-0.4em
                           {\cal H}\kern-0.5em \kern-0.4em
                           {\cal H}\kern0.1em }
\newcommand {\obfH}     {\; \overline{\cal H}\kern-0.5em \kern-0.4em 
                           \overline{\cal H}\kern-0.5em \kern-0.4em 
                           \overline{\cal H}\kern0.1em }
%
%---- PARTICELLE
\def \b             {{\mathrm b}}
\def \t             {{\mathrm t}}
\def \c             {{\mathrm c}}
\def \d             {{\mathrm d}}
\def \u             {{\mathrm u}}
\def \e             {{\mathrm e}}
\def \q             {{\mathrm q}}
\def \g             {{\mathrm g}}
\def \p             {{\mathrm p}}
\def \s             {{\mathrm s}}
\def \n             {{\mathrm n}}
\def \l             {\ell} 
%\def \f             {{\mathrm f}} 
\def \f             {{f}} 
\def \D             {{\mathrm D}}
\def \K             {{\mathrm K}}
\def \Z             {{\mathrm Z}}
\def \W             {{\mathrm W}}
\def \S             {{\mathrm S}}
\def \N             {{\mathrm N}}
\def \L             {{\mathrm L}}
\def \R             {{\mathrm R}}
%
%---- SUSY
\newcommand {\dm}         {\Delta m}
\newcommand {\dM}         {\Delta M}
\newcommand {\ldm}        {\mbox{``low $\dm$''}}
\newcommand {\hdm}        {\mbox{``high $\dm$''}}
\newcommand {\nnc}        {{\overline{\mathrm n}_{95}}}
\newcommand {\snc}        {{\overline{\sigma}_{95}}}
\newcommand {\susy}       {{supersymmetry}}
\newcommand {\susyc}      {{supersymmetric}}
\newcommand {\aj}         {\mbox{\sf AJ}}
\newcommand {\ajl}        {\mbox{\sf AJL}}
\newcommand {\llh}        {\mbox{\sf LLH}}
%
%---- SPARTICELLE
\newcommand {\rpc}     {{\rm RPC}}
\newcommand {\rpv}     {{\rm RPV}}
\newcommand {\sfe}     {{\tilde{f}}}
\newcommand {\sfL}     {{\tilde{f}_{\mathrm L}}}
\newcommand {\sfR}     {{\tilde{f}_{\mathrm R}}}
\newcommand {\sfone}   {{\tilde{f}_{1}}}
\newcommand {\sftwo}   {{\tilde{f}_{2}}}
\newcommand {\sneu}    {{\tilde{\nu}}}
\newcommand {\wino}    {{\mathrm{\widetilde{W}}}}
\newcommand {\bino}    {{\mathrm{\widetilde{B}}}}
\newcommand {\se}      {{\mathrm{\tilde{e}}}}
\newcommand {\seR}     {{\mathrm{\tilde{e}_{R}}}}
\newcommand {\seL}     {{\mathrm{\tilde{e}_{L}}}}
\newcommand {\st}      {{\mathrm{\tilde{\tau}}}}
\newcommand {\stR}     {{\mathrm{\tilde{\tau}_{R}}}}
\newcommand {\stL}     {{\mathrm{\tilde{\tau}_{L}}}}
\newcommand {\stone}   {{\mathrm{\tilde{\tau}_{1}}}}
\newcommand {\sttwo}   {{\mathrm{\tilde{\tau}_{2}}}}
\newcommand {\sm}      {{\mathrm{\tilde{\mu}}}}
\newcommand {\smR}     {{\mathrm{\tilde{\mu}_{R}}}}
\newcommand {\smL}     {{\mathrm{\tilde{\mu}_{L}}}}
\newcommand {\Sup}     {{\mathrm{\tilde{u}}}}
\newcommand {\suR}     {{\mathrm{\tilde{u}_{R}}}}
\newcommand {\suL}     {{\mathrm{\tilde{u}_{L}}}}
\newcommand {\sdo}     {{\mathrm{\tilde{d}}}}
\newcommand {\sdR}     {{\mathrm{\tilde{d}_{R}}}}
\newcommand {\sdL}     {{\mathrm{\tilde{d}_{L}}}}
\newcommand {\sch}     {{\mathrm{\tilde{c}}}}
\newcommand {\scR}     {{\mathrm{\tilde{c}_{R}}}}
\newcommand {\scL}     {{\mathrm{\tilde{c}_{L}}}}
\newcommand {\sst}     {{\mathrm{\tilde{s}}}}
\newcommand {\ssR}     {{\mathrm{\tilde{s}_{R}}}}
\newcommand {\ssL}     {{\mathrm{\tilde{s}_{L}}}}
\newcommand {\stopR}   {{\mathrm{\tilde{t}_{R}}}}
\newcommand {\stopL}   {{\mathrm{\tilde{t}_{L}}}}
\newcommand {\stopone} {{\mathrm{\tilde{t}_{1}}}}
\newcommand {\stoptwo} {{\mathrm{\tilde{t}_{2}}}}
\newcommand {\sto}     {{\mathrm{\tilde{t}}}}
\newcommand {\SQ}      {{\mathrm{\widetilde{Q}}}}
\newcommand {\STO}     {{\mathrm{\widetilde{T}}}}
\newcommand {\glu}     {{\mathrm{\tilde{g}}}}
\newcommand {\sbotR}   {{\mathrm{\tilde{b}_{R}}}}
\newcommand {\sbotL}   {{\mathrm{\tilde{b}_{L}}}}
\newcommand {\sbotone} {{\mathrm{\tilde{b}_{1}}}}
\newcommand {\sbottwo} {{\mathrm{\tilde{b}_{2}}}}
\newcommand {\sbot}    {{\mathrm{\tilde{b}}}}
\newcommand {\squa}    {{\tilde{\mathrm{q}}}}
\newcommand {\squal}   {{\tilde{\mathrm{q}}_{\rm L}}}
\newcommand {\squar}   {{\tilde{\mathrm{q}}_{\rm R}}}
\newcommand {\sqL}     {{\tilde{\mathrm{q}}_{\rm L}}}
\newcommand {\sqR}     {{\tilde{\mathrm{q}}_{\rm R}}}
\newcommand {\snu}     {{\tilde{\nu}}}
\newcommand {\snue}    {{\tilde{\nu}_{\mathrm e}}}
\newcommand {\snum}    {{\tilde{\nu}_{\mu}}}
\newcommand {\snut}    {{\tilde{\nu}_{\tau}}}
\newcommand {\neu}     {{\chi}}
\newcommand {\chap}    {{\chi^+}}
\newcommand {\cham}    {{\chi^-}}
\newcommand {\chapm}   {{\chi^\pm}}

%
%---- SUSY PARAMETRI
\newcommand {\thstop} {\mathrm{\theta_{\tilde{t}}}}
\newcommand {\thsbot} {\mathrm{\theta_{\tilde{b}}}}
\newcommand {\thsqua} {\mathrm{\theta_{\tilde{q}}}}
\newcommand {\Mcha}{M_{\chi^\pm}}
\newcommand {\Mchi}{M_\chi}
\newcommand {\Msnu}{M_{\tilde{\nu}}}
\newcommand {\tanb}{\tan\beta}
%
%---- ABBREVIAZIONI

%
%---- PROCESSI FISICI
\newcommand {\rb}    {{\rm R_{\b}}}
\newcommand {\qq}    {{\q \overline{\q}}}
\newcommand {\bb}    {{\b \overline{\b}}}
\newcommand {\ff}    {{\f \bar{\f}}}
\newcommand {\el}    {{\e ^+}}
\newcommand {\po}    {{\e ^-}}
\newcommand {\ee}    {{\e ^+ \e ^-}}
\newcommand {\gaga}  {\gamma\gamma}
\newcommand {\ggqq}  {\gamma\gamma \rightarrow \q\overline{\q}}
\newcommand {\ggtt}  {\gamma\gamma \rightarrow \tau^{+}\tau^{-}}
\newcommand {\qqg}   {\q\overline{\q}\gamma}
\newcommand {\ttg}   {\tau^{+}\tau^{-}\gamma}
\newcommand {\wenu}  {{\rm We\nu_\e}}
\newcommand {\gsZ}   {\gamma^\star\mathrm{Z}}
\newcommand {\ggh}   {\gamma\gamma\rightarrow{\mathrm{hadrons}}}
\newcommand {\ZZg}   {\mathrm ZZ^{*}/\gamma^{*}}
%
%---- VARIABILI
\newcommand {\zo}      {{z_0}}
\newcommand {\ip}      {{d_0}}
\newcommand {\thr}     {{T_{\rm thrust}}}
\newcommand {\athr}    {{\hat{\rm a}_{\rm thrust}}}
\newcommand {\acol}    {{\Phi_{\rm acol}}}
\newcommand {\acop}    {{\Phi_{\rm acop}}}
\newcommand {\acopt}   {{\Phi_{\rm acop_T}}}
\newcommand {\thpoint} {\theta_{\rm point}}
\newcommand {\thscat}  {\theta_{\rm scat}}
\newcommand {\etwelve} {E^{\, \bowtie  12\degrees}_z}
\newcommand {\ethirty} {E^{\, \bowtie  30\degrees}_z}
\newcommand {\eiso}[1] {E^{\, \triangleleft 30\degrees}_{#1}}
\newcommand {\ewedge}  {E(\phi_{\vec{p}_{\rm miss}}\pm 15\degrees)}
\newcommand {\evis}    {E_{\rm vis}}
\newcommand {\emis}    {E_{\rm miss}}
\newcommand {\mvis}    {M_{\rm vis}}
\newcommand {\mmis}    {M_{\rm miss}}
\newcommand {\mhad}    {M^{\rm ex \, \ell_1}_{\rm vis}}
\newcommand {\mhadtwo} {M^{\rm ex \, \ell_1\ell_2}_{\rm vis}}
\newcommand {\ehad}    {E^{\rm NH}_{\rm vis}}
\newcommand {\epho}    {E^{\gamma}_{\rm vis}}
\newcommand {\echa}    {E^{\rm ch}_{\rm vis}}
\newcommand {\nch}     {{N_{\rm ch}}}
\newcommand {\elept}   {E_{\rm lept}}
\newcommand {\elepone} {E_{\ell\ 1}}
\newcommand {\pvis}    {{\vec{p}_{\rm vis}}}
\newcommand {\pmis}    {{\vec{p}_{\rm miss}}}
\newcommand {\pt}      {{p_{\rm t}}}
\newcommand {\ptch}    {{p_{\rm t}^{\rm ch}}}
\newcommand {\pz}      {{p_z}}
\newcommand {\ptnoNH}  {{p_{\rm t}^{\rm ex \, NH}}}
\newcommand {\puds}    {{P_{\rm uds}}}
%
%
% no more of Christian's random capitalization!
% more of mine
\newcommand{\brchal}{\cal{B}($\PCha \rightarrow \ell\nu\PChi\ $)}
\newcommand{\M}{M_{2}}
\newcommand{\Mp}{M_{2}}
\newcommand{\sigbg}{\sigma_{\mathrm{bg}}}
\newcommand{\ww}   {\mathrm {WW}}
\newcommand{\zz}   {\mathrm Z\gamma^{*}}
\newcommand{\ewnu} {\mathrm{eW}\nu}
\newcommand{\eez}  {\mathrm {eeZ}}
\newcommand{\gagall}{{\gamma\gamma\rightarrow \ell\ell }}
\newcommand{\Pstaup}{{\widetilde{\tau}_{1}}}
\newcommand{\Pstaul}{{\widetilde{\tau}_{L}}}
\newcommand{\Pstaur}{{\widetilde{\tau}_{R}}}
\newcommand{\mzero}{m_{0}}
\newcommand{\msnu}{M_{\tilde{\nu}}}
\newcommand{\mcha}{M_{\chi^{\pm}}}
\newcommand{\mchi}{M_{\chi}}
\newcommand{\mstau}{M_{{\widetilde{\tau}_{1}}}}
\newcommand{\atau}{A_{\tau}}
\newcommand{\chsnu}{\PCha \rightarrow \ell \tilde{\nu}}
\newcommand{\chstau}{\PCha \rightarrow \tilde{\tau}_{1}\nu}
\newcommand{\chlep}{\PCha \rightarrow \ell\nu\chi}
\newcommand{\Tcsq}{\mathrm{TeV}/c^2}
% new for thesis
\newcommand{\nobs}{N_{\mathrm{obs}}}
\newcommand{\nlim}{N_{\mathrm{lim}}}
\newcommand{\Brl}{\cal{B}_{\ell}}
\newcommand{\leff} {\mathcal{L}_{\mathrm{eff}}}
\newcommand{\dedx}{{\mathrm{d}}E/{\mathrm{d}}x}
\newcommand{\chtau}{\PCha \rightarrow \tau\nu\chi}
\newcommand{\ssqtw}{\sin^{2}\theta_{\mathrm W}}
%\newcommand{\PSql}{\tilde{\mathrm q}_L}
%\newcommand{\PSqr}{\tilde{\mathrm q}_R}
%\newcommand{\PSq1}{\tilde{\mathrm q}_1}
%\newcommand{\PSq2}{\tilde{\mathrm q}_2}
%\newcommand{\ww}{{\mathrm WW}}
%\newcommand{\zz}{{\mathrm Z\gamma^{*}}}
%\newcommand{\eez}{{\mathrm eeZ}}
\newcommand{\nnz}{{\mathrm \nu\bar{\nu}Z}}
% added by bill
\def \ggll    {\gamma\gamma \rightarrow \ell^{+}{\ell}^{-}}
\def \tautau  {\mathrm \tau^{+}\tau^{-}}
\def \ffg  {f\bar{f}(\gamma)}
\def \lll   {\ell^{+}{\ell}^{-}}
\def \ww   {\mathrm WW}
\def \zz   {\mathrm Z\gamma^{*}}
\def \znn  {\mathrm Z\nu\nu}
\def \zee  {\mathrm Zee}
\def \rts  {\sqrt{s}}
\def \mstop {m_{\tilde{\mathrm{t}}}}
\def \msnu  {m_{\tilde{\nu}}}
\def \elow   {E_{12^{\circ}}}
\def \thmiss {\theta_{P_{\mathrm{miss}}}}
\def \gev    { \, \mathrm{GeV}/\it{c}^{\mathrm{2}}}
\def \gvm    { \, \mathrm{GeV}/\it{c}}
\def \mx     {M_{\mathrm{eff}}} 
\newcommand{\neutr}{\chi}
%end fabio



%dalla mia pretesi

%\def \X             {\mathrm X} 
%\def \V             {\mathrm V} 
\def \Zcc           {\Z \to \c \bar{\c} }
\def \Zbb           {\Z \to \b \bar{\b} }
\def \decDS         {\D^{*+} \to \D^0 \pi^+}
\def \decsDS        {\D^{*+} \to \D^0 \pi^+_s}
\def \deckp         {\D^{0} \to \K^- \pi^+}
\def \deckppp       {\D^{0} \to \K^- \pi^+ \pi^+ \pi^-}
\def \deckpp        {\D^{0} \to \K^- \pi^+ \pi^0}
\def \deckpS        {\D^{0} \to \K^- \pi^+ (\pi^0)}
\def \decskp        {\D^{*+} \to \pi^{+}_{s} \K^- \pi^+}
\def \decskppp      {\D^{*+} \to \pi^{+}_{s} \K^- \pi^+ \pi^+ \pi^-}
\def \decskpp       {\D^{*+} \to \pi^{+}_{s} \K^- \pi^+ \pi^0}
\def \decskpS       {\D^{*+} \to \pi^{+}_{s} \K^- \pi^+ (\pi^0)}
\def \epsc          {\varepsilon_{\c}}
\def \epsb          {\varepsilon_{\b}}
\def \pctod         {P_{\c \to \D^*}}
\def \pbtod         {P_{\b \to \D^*}}
%\def \R             {{\mathrm R}}
\def \Gbb           {\Gamma_{\b\bar{\b}}}
\def \Gcc           {\Gamma_{\c\bar{\c}}}
\def \Gh            {\Gamma_{\mathrm h}}





\pagestyle{plain}
\pagenumbering{gobble}
\noindent

\noindent  {\Large\sf\textbf{ANDREA FIASCHI} PhD Research Project}
\bigskip

\noindent {\bf Abstract} My PhD project focuses both on detector R\&D and data analysis. I would keep on studying the Serial Powering for the CMS Inner Tracker for Phase-2 within the larger scope of a system-wide test, exploiting the knowledge acquired during my degree thesis work; in the same time, I aim to analyze Phase-1 RunII data set (collected in 2017-2018) to investigate for HH production processes to learn how detector performance impacts on physics; finally, I would combine both aspects into running Monte Carlo simulation in the upgraded HL-LHC scenario to evaluate the improved CMS potential in the Higgs boson pair production channels, that will greatly benefit of the detector upgrade.
%\vspace{-.3cm}
\bigskip

\noindent {\bf The LHC and CMS experimental schedule}
As of today, CMS has collected $\sim 150\ifb$ of pp collisions data at an energy between 7 and 13$\TeV$. In 2021, after two years of shutdown, the energy will increase to $14\TeV$ and the total integrated luminosity should reach $\sim 300\ifb$ by 2023. The data set collected so far has already allowed the discovery of the Higgs boson and is now being used to measure its properties.

However, a change of pace will follow with the High Luminosity phase (HL-LHC) from 2026 to 2039, when the $10\times$ larger integrated luminosity, around  $3000-4500\ifb$, will enable high precision measurements and the
exploration of the multi-TeV scale by searches for particles with high masses and by observation of unexpected phenomena, e.g. long-lived massive charged particles. %, that would be regarded as hints of new physics. % Additionally, models with an extended Higgs sector may manifest themselves in additional Higgs bosons and other exotic new particles.
 
Another path for discoveries is the high precision measurements of known processes to look for predictions deviations. For example, many new physics models predict $\cal{O}(\%)$ alterations from Standard Model of the Higgs boson couplings and Higgs boson pair production (HH) allows the Higgs boson self-coupling to be directly probed. 
%By precisely measuring the properties of the Higgs boson, these predictions can be tested.

To reach these goals the CMS experiment will be massively upgraded and, most important among many improvements, the tracker detector will be completely replaced. This new apparatus will feature several frontier technologies to comply with challenging requirements. In particular the Inner Tracker %, the innermost part of the new tracker detector,
will implement a Serial Powering scheme, that has never been used and is a crucial ingredient for Phase-2. %: radiation tolerance, increased granularity, improved two-track separation, reduced material in the tracking volume, robust pattern recognition and extended tracking acceptance.

\subsection*{PhD research project}
%\vskip -.2cm
%\subsubsection*{Analysis on current CMS data set: Higgs boson pair production}
%\vskip -.2cm
\begin{description}[style=unboxed,leftmargin=.2cm]
\item[Analysis on current CMS data set: Higgs boson pair production.] Searches for Higgs boson pair production have been performed by the ATLAS and CMS experiments using LHC pp collision data. These include searches for Beyond Standard Model production as well as more targeted searches for production with Standard Model-like kinematics in $\sqrt{s} = 8\TeV$ and $13\TeV$ data. 
In particular the CMS Firenze group is involved in the important analyses of resonant and non-resonant Higgs boson pair production. %, presented by CMS in~\cite{Analysis}, based on LHC pp collision data at $\sqrt{s}=13$ TeV collected by the CMS experiment, corresponding to an integrated luminosity of $35.9\ifb$. 
%The analysis targets $\mathrm{X \rightarrow HH}$. %where one of the H decays into $\mathrm{b\overline{b}}$, and the other into Z($\ell\ell$) Z($\nu\nu$) or W($\ell\nu$) W($\ell\nu$), where $\ell$ is either an electron, a muon, or a tau lepton that decays leptonically. 
%The search presented in~\cite{Analysis} by CMS Collaboration (that includes the CMS Firenze group) is based on LHC pp collision data at %$\sqrt{s}=13$ TeV collected by the CMS experiment, corresponding to an integrated luminosity of 35.9 $fb^{-1}$. 
%Nonresonant Higgs boson pair production (HH) can be used to directly study the Higgs boson self-coupling. 
 In the Standard Model, the destructive interference between different diagrams makes the observation of HH production extremely challenging.
%, even in the most optimistic scenarios of energy and integrated luminosity at the future HL-LHC. 
Indirect effects at the electroweak scale arising from Beyond Standard Model phenomena at a higher scale can be parameterized in an effective field theory framework by introducing coupling modifiers for the Standard Model parameters involved in HH production. %, namely $k_{\lambda}=\lambda /\lambda_{SM} $ for the Higgs boson self-coupling $\lambda$ and $k_t= y_t /y_{t_{SM}}$ for the top quark Yukawa coupling $y_t$. 
 Such modifications of the Higgs boson couplings could enhance Higgs boson pair production to rates observable with the Phase-1 data set.  
\begin{siderules}
\noindent {\sf In my PhD activity I will analyze RunII data for HH final states, a very interesting physics process, to gain experience with data analysis of a running experiment.}
\end{siderules}
\end{description}

\begin{description}[style=unboxed,leftmargin=.2cm]
\item[Studies of Serial Powering for Phase-2 Inner Tracker upgrade.] Extended tracking acceptance, high granularity, reduced space and the need of minimizing the material in the tracking volume brought to chose a Serial Powering scheme for the Inner Tracker, instead of parallel powering or the use of \mbox{DC-DC} converters (that cannot withstand the high level of radiation expected for Phase-2).
%, $2.3 \cdot 10^{16} n_{\mathrm{1 MeV eq}}/cm^2$, 1.2 Grad).
A dedicated ``ShuntLDO'' circuit~\cite{SLDO} (a regulator \textit{Low Drop Out} coupled with a Shunt) handles the supply locally on each ReadOut Chip (ROC). The first ROC prototype, implemented in $65\nm$ CMOS technology, is already available, RD53A~\cite{RD53A}. 
Up to now the validation of the ShuntLDO performances has been evaluated stand alone or with few ROCs connected together. 
These studies have been also performed in Firenze INFN Laboratory and were part of my Master's Degree Thesis~\cite{tesi}. Thanks to this background I will thoroughly study and characterize serial power chains based on prototype multi-chip modules. Among many others, the analysis of the failure modes and the design of the HV distribution are of particular importance to arrive, in 2020, to the final pixel module project.
\begin{siderules}
\noindent {\sf In my PhD activity I will take over the next crucial step for Serial Powering development, i.e. the test of an extended serial power chain within the larger and more ambitious scope of a full system test crucial for the design, development and construction of the new Inner Tracker system. %For 2019 and 2020 these are activities already planned in Firenze.
}
%Also, for the Outer Tracker there will be improvement in the powering scheme.
\end{siderules}
\end{description}

\begin{description}[style=unboxed,leftmargin=.2cm]
\item[Simulation based Phase-2 physics studies: \bm{$\mathrm{HH \rightarrow b\overline{b}b\overline{b}}$}.] At HL-LHC %not only the luminosity will change but also 
the entire CMS experiment will be upgraded, hence the response of the detector will change and it is crucial to study future analyses by running simulation studies to shape and optimize event reconstruction. In particular the upgraded Phase-2 detector is designed to cope with the challenging environment of \mbox{HL-LHC} and the performance of pileup mitigation, b-tagging, tau-tagging, photon identification efficiencies, and mass resolutions are instrumental to perform physics studies and high precision measurements. 
%== per il mio phd voglio occuparmi di analisi con stati finali HH perché ritengo il caso di fisica molto interessante e per guadagnare esperienza con l'analisi di un esperimento running [bla, bla, bla]...
Better performances will be achieved also thanks to the detector I'm contributing to and to Serial Powering in particular. By using simulations, it would be very interesting to evaluate the impact of these detector improvements on a physics analysis similar to the one I'm planning to study on current CMS data. %I aleady experienced Monte Carlo studies in Bachelor’s Degree Thesis.

In fact the observation of the HH processes will greatly benefit of the upgraded tracker thanks to a better momentum resolution and excellent b-tagging capabilities, especially in the $\mathrm{HH \rightarrow b\overline{b}b\overline{b}}$ final states that profit of the largest branching ratio. 
The increased acceptance of the tracker above $|\eta|>2.4$ provides an improvement in the acceptance of the four jets coming from the HH decay resulting in larger signal efficiency. Overall, also thanks to the much better performance of the b-tagging algorithms, the efficiency of identifying four b jets per event increases, on average, by 50-70\% compared to the RunII analysis~\cite{TDR}.
%Given the all-jets signature of the final state the presence of an average of 140–200 pileup events, with an increased level of activity and energy flow, presents a serious challenge. Thus PU mitigation is an important consideration for this analysis. 
\begin{siderules}
\noindent {\sf In my PhD activity I will use Monte Carlo simulations I'm already experienced with~\cite{TesiTriennale} to study the benchmark channel $\mathrm{HH \rightarrow b\overline{b}b\overline{b}}$. 
It is very interesting to evaluate the impact of the upgraded detector in comparison with similar Phase-1 RunII analyses.} 
\end{siderules}
\end{description}

%because the signature with four b quarks in the final state offers the highest branching fraction

%Higgs boson pair production is the most direct way to study the scalar potential of the SM
%Higgs boson and the nature of electroweak symmetry breaking. The observation of this process
%is an important objective of the HL-LHC program. Higgs pair production can occur through
%its trilinear self-coupling or through a box diagram. In the SM the two processes interfere de-
%structively, resulting in a near minimal Higgs boson pair production cross section.
%A large amount of integrated luminosity is required in order to observe this extremely rare
%process. Projections for the HL-LHC indicate that SM di-Higgs production can be observed by
%CMS with a significance of 1.9 standard deviations by combining analyses in various channels
%and with a data set corresponding to an integrated luminosity of 3000 fb − 1
%The signature with four b quarks in the final state (HH → bbbb) offers the highest branching
%fraction. The main challenge of this search is to distinguish the signal of four final state bottom
%quarks that hadronize into jets (b jets) from the copious multijet background described by quan-
%tum chromodynamics (QCD). We address this challenge by suitable event selection criteria that
%include dedicated b jet identification techniques and a model of the multijet background that is
%validated in data control regions. Given the low cross section of the signal process, maintaining
%a high signal efficiency is a critical requirement.



%Previous studies on the double Higgs production at the HL-LHC have been performed in various final states including $\mathrm{b\overline{b}\gamma\gamma}$, $\mathrm{b\overline{b}\tau^+\tau^-}$, $\mathrm{b\overline{b}W^+W^-}$ and $\mathrm{b\overline{b}b\overline{b}}$. 
%Among these, despite its lower signal rate, the $\mathrm{b\overline{b}\gamma\gamma}$ (BR $\simeq$ 0.268\%) decay channel  is  the  cleanest  and  most  thoroughly  studied  in  literature.  
%Apparently,  switching to  the $\mathrm{b\overline{b}\tau^+\tau^-}$ (BR $\simeq$ 7.31\%)  decay  channel  may  look  promising,  as  the  cross  section increases  by  a  factor  of  27.7  compared  to  the $\mathrm{b\overline{b}\gamma\gamma}$ decay  channel. 
%However, the signal rate is penalized by a series of factors (sub-branching ratios of the $\mathrm{\tau^+\tau^-}$ system, tau background processes, tagging efficiency...). 
%In my PhD activity I will investigate the feasibility of studies in $\mathrm{b\overline{b}\tau^+\tau^-}$ with the upgrade of phase-2 and the advantages coming from a combined analysis of the double Higgs production via gluon fusion in the $\mathrm{b\overline{b}\gamma\gamma}$ and $\mathrm{b\overline{b}\tau^+\tau^-}$ decay channels at the HL-LHC.

%Previous studies on the double Higgs production at the HL-LHC have been performed
%in various final states including bbγγ[11, 24–33],bbτ+τ−[31, 34, 35],b ̄bW+W−[36, 37] andb ̄bb ̄b[38–40].  Among those, despite its lower signal rate,theb ̄bγγ(BR'0.264\%) decay channel  is  the  cleanest  and  most  thoroughly  studied  in  literature.  Apparently,  switching to  the bbτ+τ−(BR'7.31\%)  decay  channel  may  look  promising,  as  the  cross  section increases  by  a  factor  of  27.7  compared  to  theb ̄bγγ decay  channel.   However,  the  signalate is penalized by a series offactors,the first one being the sub-branching ratios of theτ+τ−system, namely 42.3\% and 45.5\% for the fully hadronicand semileptonic final states,respectively.   The  signal  rate  is  further  penalized  by  the  tau  tagging efficiency.   Finally,reconstructing theh→τ+τ−system, for instance, againstZ→τ+τ−is challenging due tothe irreducibleloss of information via invisible neutrinos [41].  As a result, the performanceof theb ̄bτ+τ−channel does not look better than the bbγγ channel [31].  The bbτ+τ− may have the potential to be at best comparable to the bbγγ. 

%Also, for the Outer Tracker there will be improvement in the 

%On the other hand, I also would evaluate, how these studies will be extended for the phase at High Luminosity. This point will be an important focus in the next years. 
%
% So, my purpose, is to run Monte Carlo simulation on final states, because in HL-LHC not only the luminosity will change but also the entire CMS experiment will be upgrade. This means, that also the detector response, acceptance granularity, passive material will be different, for this reason Monte Carlo studies will play an important role. 
 
%== accanto a questo, però, voglio anche valutare, in prospettiva, come questi studi di fisica si estendono a HL-LHC dato che questo sarà il focus dei prossimi anni e quindi voglio fare degli studi di montecarlo su stati finali ... (cosa avevamo detto con piergiulio?) perché questo serve per questo e per quest'altro [bla, bla, bla];
%These studies will be also useful for data analysis.
%== queste analisi saranno grandemente avvantaggiate dai miglioramenti del nuovo tracciatore IT (granularita, estensione, ...) che permettono di fare fisica in avanti e pile-up mitigation. Queste migliorie saranno possibili anche grazie al serial powering, che, guarda caso, ho studiato per la tesi di laurea, e che quindi voglio continuare a seguire nella fase successiva (system test) in modo da contribuire direttamente al disegno e allo sviluppo e alla costruzione del nuovo apparato acquisendo una conoscenza approfondita dello stesso che poi mi servirà per le analisi di fisica per le quali lo utilizzerò."

%First part of the doctoral project is a Serial Powering activity, for the final part of 2018 includes  Tests with bare RD53A on Single Chip Card (SCC) and Tests with single RD53A modules on SCC.
%For 2019 are expected the first tests with 1x Serial Powering chain on bench (general tests) and 1x Serial Powering chains on ladder prototype + test. More in general powering scheme concepts will be tested both in laboratory and on beam from 2019 to 2021 to evaluate behaviour with respect to performance figures (reliability, induced noise, voltage stability, failure handling).
%Furthermore the IT unit (Firenze) and CAEN, leader company in electronics for high energy physics, are participating into the joint projects, funded by EU through Regione Toscana, NEXTLITE (2018-2019) and PRIMIS (2019-2020) dedicated to the development of large-scale, LV and HV integrated, supply systems. In particular, the objective is to test in 2021 a realistic power group with CMS-ROC full scale modules to finalize the power scheme design for the experiment.
% 
%
%On the other hand with the data taking of 2017 and 2018 is feasible the analysis for the search for resonant and nonresonant Higgs boson pair production in the $b \overline{b}l\nu l\nu$ final state in proton-proton collision at $\sqrt{s}=13$ TeV with a better statistics, expanding beyond mass range. Also a simulation of data take after Long Shutdown 3 (LS3) can be performed, with the upgrade of CMS not only the luminosity will change but also the response of the detector (tracking acceptance, track separation ...).


\vspace{.3cm}
%\subsection*{Expected results}
%\vskip -.3cm
%{\bf Questo non so se serve}
%
%The project is based on tests and searches in which CMS Firenze group is involved already. Several innovative aspects and technologies will be addressed in the present project. The new CMS Tracker will be significantly different, in many key features, from similar detectors currently operational. Gaining early experience on Serial Powering as well on data analysis (resonant and nonresonant Higgs boson pair production) and Monte Carlo simulation, is important to learn how to exploit the full potential of the new tracker. This acquired competence will be useful for the final experiment that, from day one in 2026, must operate at the maximum of its capability.
%
%\vspace{-.3cm}
%\renewcommand{\bibname}{References} 
%\subsection*{References}
\begin{small}
\begin{thebibliography}{1}
%\bibitem{Analysis}
%%\bibitem{Sirunyan:2017guj}%
%  A.~M.~Sirunyan {\it et al.} [CMS Collaboration], ``Search for resonant and nonresonant Higgs boson pair production in the $ \mathrm{b}\overline{\mathrm{b}}\mathit{\ell \nu \ell \nu } $ final state in proton-proton collisions at $ \sqrt{s}=13 $ TeV'',  JHEP {\bf 1801} (2018) 054,  doi:10.1007/JHEP01(2018)054, [arXiv:1708.04188 [hep-ex]].

\bibitem{SLDO} M.~Karagounis, D.~Arutinov, M.~Barbero et al. ``An Integrated Shunt-LDO Regulator for Serial Powered Systems''. In: \textit{Proc. of the European Solid-State Device Conference,
ESSCIRC 2009 (2009)}, pp. 276–279. doi: 10.1109/ESSCIRC.2009.5325974.

\bibitem{RD53A} M.~Garcia-Sciveres, ``The RD53A Integrated Circuit''
CERN-RD53-PUB-17-001 (2017), \url{https://cds.cern.ch/record/2287593}.

\bibitem{tesi} A. Fiaschi, ``Development of an innovative powering scheme for CMS Inner Tracker for High Luminosity phase of LHC'', Master Degree Thesis (2018).

\bibitem{TDR} CMS Collaboration, ``The Phase-2 Upgrade of the CMS
  Tracker'', 2017, CERN-LHCC-2017-009; CMS-TDR-014.

\bibitem{TesiTriennale} A. Fiaschi, ``Monte Carlo simulation for ${\mathrm H} \rightarrow \mathrm{\W \W}\rightarrow 2\ell + 2\nu$ study at LHC and comparison with CMS data'', \url{www.infn.it/thesis/PDF/getfile.php?filename=11405-Fiaschi-triennale.pdf}, Bachelor’s Degree Thesis (2016).

\end{thebibliography}
\end{small}

\end{document}
%1) This proposal is based on the participation of several young researchers alongside senior staff. This is of paramount importance to build
%up the expertise needed to operate, run and maintain, for over 10 years after 2026, the new CMS Tracker that will be shaped out also from
%the present project. This project is an investment, in an international and competitive environment, on the future of science in general and
%particle physics in particular through the young generations.
%2) CERN is the world leading laboratory for high energy physics at colliders and the US is the largest contributor to CERN accelerators and
%detectors. For example US-CMS represents almost 30% of the entire CMS collaboration and it is participating accordingly to the detector
%upgrade. The proposal therefore builds on the existing scientific collaboration between Italy and the US within CMS and enables further
%exchange between the two countries.
%3) The involvement in this project of CAEN as industrial partner will facilitate the dissemination of the acquired technological knowledge,
%namely on power supply technologies, not only within the niche field of high energy physics, but also in nuclear and health science
%applications that, on the contrary, are of greatest importance for everybody daylife.
%
%WP1: the validation of the pixel detector prototypes will proceed by testing in beam samples irradiated at different fluences corresponding to
%10-years of HL-LHC operating conditions. In 2019 RD53A-ROC modules will be tested to validate and optimize the cell layout for both
%planar and 3D sensor technologies. The results will then be used to design sensors for the CMS-ROC to be then tested as well with a
%similar approach through 2020 and 2021 to arrive to the final sensor design.
%Thanks to the high beam intensity and to the tracking provided by the FNAL telescope, the high rate tests allow us to study in detail hit and
%charge collection efficiency, whose outcome will serve as input for sensor optimization. The IT unit (Firenze) is responsible for developing
%the Si-detectors within the framework of the Advanced Hybrid Pixel Detector working package (WP7) of the Horizon 2020 - Advanced
%European Infrastructures for Detectors at Accelerators project (AIDA2020). The US unit (FNAL) will provide support for the integration,
%calibration, readout and monitoring of the devices at the beam test facility.
%
%WP2: the OTSDAQ system will be deployed to integrate the RD53A-ROC in 2019, and it will be extended to CMS-ROC in 2020. All features
%(web interface, automated calibration tasks, data and histogram sharing) will be tested and validated during the 2019 and 2020 beam tests.
%These are small scale experiments that enable to probe all aspects of a DAQ system: run control on several devices, trigger handling,
%online monitoring. In 2021, high-rate beam tests of multi-chip full size modules will aid in estimating the computing resources required by the
%OTSDAQ tasks. This is a key step to finalize the design of the DAQ boards for the experiment.
%
%WP3: powering scheme concepts will be tested both in laboratory and on beam from 2019 to 2021 to evaluate behaviour with respect to
%performance figures (reliability, induced noise, voltage stability, failure handling). The IT unit (Firenze), which had a cardinal role in
%designing the power system for the current CMS Tracker, and CAEN, leader company in electronics for high energy physics, are
%participating into the joint projects, funded by EU through Regione Toscana, NEXTLITE (2018-2019) and PRIMIS (2019-2020) dedicated to
%the development of large-scale, LV and HV integrated, supply systems. Prototypes will be deployed in FNAL beam tests. In particular, the
%objective is to test in 2021 a realistic power group with CMS-ROC full-scale modules to finalize the power scheme design for the
%experiment. The US unit has a key role in integrating such prototypes into the beam test environment, DAQ and controls.
%
%
%MILESTONES
%2019
%Jul 1: RD53A-ROC integrated in OTSDAQ (WP2)
%Dec 1: Beam test of planar and 3D rad-hard sensors with RD53A-ROC (WP1)
%2020
%Jul 1: Beam test of first CAEN prototypes (WP3)
%Oct 1: CMS-ROC integrated in OTSDAQ (WP2)
%Dec 1: Beam test of planar and 3D rad-hard sensors with CMS-ROC (WP1)
%2021
%Oct 1: Test of power supply system test (WP3)
%Dec 1: Beam test of CMS-ROC multi-chip modules (WP1)
%
%Starting from 2026, the HL-LHC will achieve an instantaneous luminosity of $5 \times 10 34 cm - 2 s - 1$,
%with a bunch spacing of 25 ns, as described in Chapter 1. In each bunch crossing, the CMS
%tracker will be traversed by around 6000 charged particles (inclusively, counting reconstructed
%tracks with p T > 300 MeV), produced by about 200 collisions on average, and in those challeng-
%ing conditions excellent tracking performance has to be maintained.